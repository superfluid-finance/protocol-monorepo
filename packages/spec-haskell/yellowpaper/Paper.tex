%%%%%%%%%%%%%%%%%%%%%%%%%%%%%%%%%%%%%%%%%%%%%%%%%%%%%%%%%%%%%%%%%%%%%%%%%%%%%%%%
%% Document Preamble
%%%%%%%%%%%%%%%%%%%%%%%%%%%%%%%%%%%%%%%%%%%%%%%%%%%%%%%%%%%%%%%%%%%%%%%%%%%%%%%%
\documentclass[a4paper]{article}

%% Base Folder for Inputs
\makeatletter
\def\input@path{{../tex/}}
\makeatother

%% Import Packages
\usepackage{fvextra}
\usepackage{csquotes}
\usepackage{xcolor}
\usepackage{minted}
\usepackage{amsmath}
\usepackage{enumitem}

%% Literate Haskell Macros

%%% include common formatting
\input{lhsfmt.tex}
%%% for ignoring some code
\long\def\ignore#1{}
%%% for creating a paragraph
\newcommand{\lhsparagraph}[1]{\paragraph{#1}\mbox{}\\}

%% Document Meta Data

\title{Denotational Semantics of General Payment Primitives, and Its Payment System}

\author{\\
    Miao, ZhiCheng\\
    Co-Founder, Superfluid Finance\\
    miao@superfluid.finance
}

\usepackage[
    backend=biber,
    style=alphabetic,
]{biblatex}
\addbibresource{../tex/Biblio.bib}

%%%%%%%%%%%%%%%%%%%%%%%%%%%%%%%%%%%%%%%%%%%%%%%%%%%%%%%%%%%%%%%%%%%%%%%%%%%%%%%%
%% Document Body
%%%%%%%%%%%%%%%%%%%%%%%%%%%%%%%%%%%%%%%%%%%%%%%%%%%%%%%%%%%%%%%%%%%%%%%%%%%%%%%%
\begin{document}
\maketitle

\begin{abstract}
    \begin{center}
        (Some distant lyrics as the placeholder of the actual abstract)

        Money, money, money

        Must be funny

        In the rich man's world

        Money, money, money

        Always sunny

        In the rich man's world

        \

        Ah, all the things I could do

        If I had a little money

        It's a rich man's world

        It's a rich man's world
    \end{center}
\end{abstract}

\section{Introduction}

It should be fair to say, every aspect of money are controversial: the nature of money, the value of money, money and
banking, and monetary reconstruction. Two major schools of thoughts about theory of money are the \textit{Austrian
    school} \cite{von2013theory} and the \textit{Chicago school} \cite{friedman1989quantity}. That is before the
appearance of Internet-era version of monetary reconstruction, broadly defined as cryptocurrency, which challenges
theories of money further and demands their updates \cite{ammous2018can} \cite{hardle2020understanding}.

But when it comes to the function of money, it is less controversial. According to Von Mises,

\begin{quotation}
The function of money is to facilitate the business of the market by acting as a common medium of
exchange. \footfullcite[][Part One, Chapter I, § 1, p1]{von2013theory}
\end{quotation}

How do different forms of money perform this function, especially in the information age, when electronic forms of money
are increasingly used?

This yellow paper adds a new controversy to money, that is to present a survey challenging the preconceived notion of
how money can perform its function of medium of exchange using computer science.

In Part I, we shall first explore the foundation for the survey. Here we present a formal definitions of payment system,
and various models in the system. We then select a few relevant approaches used in computer science useful for defining
formal specification of payment system.

One of the approaches is denotational semantics, which is used in Part II of the paper to define the general payment
primitives. Along with the denotational semantics, a restatement of it in \textit{Haskell programming language} is also
included.

In Part III, a reference implementation of the general payment primitives and its payment system called
\textit{Superfluid Money} is introduced.

In Part IV, some possible further investigations are included for future study purpose.

\newpage
\part{Foundation}
\newpage

\section{Nature of Money}

Nature of money.

\section{Money Distribution}

Most of the existing money schemes are of this category, where a money unit completely disregards its context and hence
its liquidity value and bearer are static until the next money redistribution occurs.

Here are the type definitions of these money schemes defined in \cite{buldas2021unifying}:

TODO...

\subsection{A New Unifying Theory}

\input{MoneyDistributionConcepts.tex}

\section{Forms of Medium}

\section{Payment Models}

\subsection{Discrete Payments}

\subsection{Continuous Payments}

In contract, under the contextual money schemes, the liquidity value and maybe even bearer of a money unit can be a
function of context too.

TODO...

\section{Related Paradigms}

\subsection{Constraint Programming}

\subsection{Functional Reactive Programming}

\subsection{Denotational Semantics}

\section{Generalized Denotational Payment Semantics}

Here is the specification of denotational semantics that support both discrete and continous payments.

Here is the list of acronyms used:

\begin{itemize}
    \item \textbf{MD} is \textit{Money Distribution}.
    \item \textbf{MU} is \textit{Monetary Unit}.
    \item \textbf{RTB} is \textit{Real Time Balance}.
\end{itemize}

\subsection{Generalized Payment Model}

\begin{equation}\label{sem_transfer}
    \begin{split}
        [\![MD\ mu\ t\ rtb]\!] = mu \rightarrow t \rightarrow rtb
    \end{split}
\end{equation}

\subsection{Monoidial Payment Model}

\begin{equation}\label{sem_mzero}
    [\![\emptyset]\!] = \lambda\ mu\ t\ \rightarrow 0
\end{equation}

\begin{equation}\label{sem_mappend}
    [\![mda \oplus\ mdb]\!] = \lambda\ mu\ t\ \rightarrow
    [\![mda]\!]\ mu\ t\ +\ [\![mdb]\!]\ mu\ t
\end{equation}

\subsection{Payment Premitives}

\begin{equation}\label{sem_transfer}
    \begin{split}
        [\![&transfer\ from\ to\ amount\ md]\!] = [\![md]\!]\ \oplus \\
        (\lambda\ case&\ mu\ t \\
        &|\ from = mu \rightarrow -amount \\
        &|\ to   = mu \rightarrow amount \\
        &|\ otherwise \rightarrow 0 \\
        )
    \end{split}
\end{equation}

\begin{equation}\label{sem_updateConstantFlow}
    \begin{split}
        [\![&updateConstantFlow\ from\ to\ flowRate\ t'\ md]\!] = [\![md]\!]\ \oplus \\
        (\lambda\ case&\ mu\ t \\
        &|\ from = mu \rightarrow -flowRate * (t - t') \\
        &|\ to   = mu \rightarrow flowRate  * (t - t') \\
        &|\ otherwise \rightarrow 0 \\
        )
    \end{split}
\end{equation}

\begin{equation}
    unit :: sub \rightarrow subs \rightarrow Int
\end{equation}

\begin{equation}
    proportion\ sub\ subs = {{ unit\ sub\ subs } \over { \displaystyle \sum_{x \in subs} unit\ x\ subs }}
\end{equation}

\begin{equation}\label{sem_distributeProportionally}
    \begin{split}
        [\![&distributeProportionally\ pub\ subs\ amount\ md]\!] = [\![md]\!]\ \oplus \\
        (\lambda\ case&\ mu\ t \\
        &|\ pub = mu \rightarrow -amount \\
        &|\ \exists sub \in subs\ { sub = mu } \rightarrow amount * (proportion\ mu\ subs) \\
        &|\ otherwise \rightarrow 0 \\
        )
    \end{split}
\end{equation}

\begin{equation}\label{sem_distributeProportionally}
    \begin{split}
        [\![&updateProportionalDistributionConstantFlow\ pub\ subs\ flowRate\ t'\ md]\!] = [\![md]\!]\ \oplus \\
        (\lambda\ case&\ mu\ t \\
        &|\ pub = mu \rightarrow -flowRate * (t - t') \\
        &|\ \exists sub \in subs\ { sub = mu } \rightarrow flowRate * (t - t') * (proportion\ mu\ subs) \\
        &|\ otherwise \rightarrow 0 \\
        )
    \end{split}
\end{equation}

\newpage
\part{Superfluid Money}
\newpage

\section{Real Time Balance}

\section{Sub Systems}

\section{Agreement Sub System}

\section{Type of Agreements}

\subsection{Transferable Balance Agreement}

\subsection{Constant Flow Agreement}

\subsection{Distribution Agreement}

\subsection{Decaying Flow Agreement}

\section{Solvency Sub Systems}

\subsection{Buffer Based Solvency Systems}

\section{Type Of Money Units}

\newpage

\printbibliography{}

\end{document}
